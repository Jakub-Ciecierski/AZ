
%%%%%%%%%%%%%%%%%%%%%%% file typeinst.tex %%%%%%%%%%%%%%%%%%%%%%%%%
%
% This is the LaTeX source for the instructions to authors using
% the LaTeX document class 'llncs.cls' for contributions to
% the Lecture Notes in Computer Sciences series.
% http://www.springer.com/lncs       Springer Heidelberg 2006/05/04
%
% It may be used as a template for your own input - copy it
% to a new file with a new name and use it as the basis
% for your article.
%
% NB: the document class 'llncs' has its own and detailed documentation,  see
% ftp://ftp.springer.de/data/pubftp/pub/tex/latex/llncs/latex2e/llncsdoc.pdf
%
%%%%%%%%%%%%%%%%%%%%%%%%%%%%%%%%%%%%%%%%%%%%%%%%%%%%%%%%%%%%%%%%%%%


\documentclass[runningheads, a4paper]{llncs}



\usepackage[linesnumbered,ruled]{algorithm2e}

\usepackage[margin=0.5in]{geometry}
\usepackage{amssymb}
\setcounter{tocdepth}{3}
\usepackage{graphicx}
\usepackage{amssymb}% http://ctan.org/pkg/amssymb
\usepackage{pifont}% http://ctan.org/pkg/pifont
\newcommand{\cmark}{\ding{51}}%
\newcommand{\xmark}{\ding{55}}%

\usepackage{floatrow}
% Table float box with bottom caption, box width adjusted to content
\newfloatcommand{capbtabbox}{table}[][\FBwidth]
\usepackage{caption}
\usepackage{subcaption}
\captionsetup{compatibility=false}

\usepackage{url} % for bibliograpy links
\urlstyle{same}  % (for bibliography links
\usepackage{float} % To force image to stand still

\usepackage{amsmath,amssymb}

\usepackage{color}

\usepackage{array}
\newcolumntype{P}[1]{>{\centering\arraybackslash}p{#1}}





\usepackage{xcolor} % Colors

\usepackage{tikz}
\usepackage{keyval}
\usepackage{pgfplots} % Bar charts

\usepgfplotslibrary{groupplots}

%\usepackage{groupplots}
%\usepgfplotslibrary{groupplots}
\pgfplotsset{compat=newest}
   
   
%
% Define bar chart colors
%
\definecolor{bblue}{HTML}{4F81BD}
\definecolor{rred}{HTML}{C0504D}
\definecolor{ggreen}{HTML}{9BBB59}
\definecolor{ppurple}{HTML}{9F4C7C}


\usepackage{amssymb}
\setcounter{tocdepth}{3}
\usepackage{graphicx}

\newcommand{\keywords}[1]{\par\addvspace\baselineskip
\noindent\keywordname\enspace\ignorespaces#1}







%%%%%%%%%%%%%%%%%%%%%%%%%%%%%%%%%%%%%%%%%
%           START OF COMMANDS
%%%%%%%%%%%%%%%%%%%%%%%%%%%%%%%%%%%%%%%%%


%%%%%%%%%%%%%%%%%%%%%%%%%%%%%%%%%
% Save results to be later used in addBarPlotResults
% #1: Train Small
% #2: Train Big
% #3: Train All
% #4: Test
%%%%%%%%%%%%%%%%%%%%%%%%%%%%%%%%%
\newcommand{\saveResultsFirst}[4]{
    \def\TrainSmallFirst{#1}%
    \def\TrainBigFirst{#2}%
    \def\TrainAllFirst{#3}%
    \def\TestFirst{#4}%
}

%%%%%%%%%%%%%%%%%%%%%%%%%%%%%%%%%
% Save results to be later used in addBarPlotResults
% #1: Train Small
% #2: Train Big
% #3: Train All
% #4: Test
%%%%%%%%%%%%%%%%%%%%%%%%%%%%%%%%%
\newcommand{\saveResultsSecond}[4]{
    \def\TrainSmallSecond{#1}%
    \def\TrainBigSecond{#2}%
    \def\TrainAllSecond{#3}%
    \def\TestSecond{#4}%
}

%%%%%%%%%%%%%%%%%%%%%%%%%%%%%%%%%
% Adds Results to Bars labeled by #1, #2
%
% Label #1 is used for saveResultsFirst
% Label #2 is used for saveResultsSecond
%
%%%%%%%%%%%%%%%%%%%%%%%%%%%%%%%%%
\newcommand{\addBarPlotResults}[2]{
    % TRAINING |w| < C    
    \addplot[style={bblue, fill=bblue, mark=none}]
    coordinates 
    {
        (#1,  \TrainSmallFirst)
        (#2,  \TrainSmallSecond)
    };
    
    % TRAINING |w| > C
    \addplot[style={rred, fill=rred, mark=none}]
    coordinates 
    {
        (#1,  \TrainBigFirst)
        (#2,  \TrainBigSecond)
    };
    
    % TRAINING ALL
    \addplot[style={ggreen, fill=ggreen, mark=none}]
    coordinates 
    {
        (#1,  \TrainAllFirst)
        (#2,  \TrainAllSecond)
    };
    
    % TESTING
    \addplot[style={ppurple, fill=ppurple, mark=none}]
    coordinates 
    {
        (#1,  \TestFirst)
        (#2,  \TestSecond)
    };
}



%%%%%%%%%%%%%%%%%%%%%%%%%%%%%%%%%
% Adds needed style to pgfplot
%%%%%%%%%%%%%%%%%%%%%%%%%%%%%%%%%
    \newcommand{\setPlotStyle}[0]{
        \pgfplotsset{xticklabel style={text width=2em, align=center}}
    }

%%%%%%%%%%%%%%%%%%%%%%%%%%%%%%%%%
% Starts groupplot with given style
%%%%%%%%%%%%%%%%%%%%%%%%%%%%%%%%%
\newcommand{\beginGroupPlot}[0]{
    \begin{groupplot}[
        group style={group size= 2 by 2,  vertical sep=50pt},
        height = 5.5cm, 
        ybar=4*\pgflinewidth, 
        ymin=0, 
        ymax=1,
        width  = 0.37*\textwidth, 
        enlarge x limits={abs=0.85cm}, 
        major x tick style = transparent, 
        ymajorgrids = true, 
        symbolic x coords={
            C=4,
            C=4 Q=3,
            C=4 Q=4,
            C=4 Q=5,
            C=4 Q=6,
            C=4 Q=7,
            C=4 Q=8,
            C=4 Q=9,
            C=4 Q=10,
            C=4 Q=11,
            C=4 Q=12,
            C=4 Q=13,
            C=4 Q=14,
            C=4 Q=15,
            C=5,
            C=5 Q=3,
            C=5 Q=4,
            C=5 Q=5,
            C=5 Q=6,
            C=5 Q=7,
            C=5 Q=8,
            C=5 Q=9,
            C=5 Q=10,
            C=5 Q=11,
            C=5 Q=12,
            C=5 Q=13,
            C=5 Q=14,
            C=5 Q=15
        },
        xtick = data, 
        scaled y ticks = false, 
        legend cell align=center, 
        legend columns=1,
        legend style={
            at={(1.45, 0.45)}, 
            anchor=north,
        }
        ]
    }%beginGroupPlot
    
%%%%%%%%%%%%%%%%%%%%%%%%%%%%%%%%%
% Sets legends for Bar plot
%%%%%%%%%%%%%%%%%%%%%%%%%%%%%%%%%
    \newcommand{\setPlotLegend}[0]{
        \legend{Training $|w|<C$,  Training $|w|>C$,  Training All,  Testing}
    }
    
%%%%%%%%%%%%%%%%%%%%%%%%%%%%%%%%%%%%%%%%%
%           END OF COMMANDS
%%%%%%%%%%%%%%%%%%%%%%%%%%%%%%%%%%%%%%%%%




\begin{document}

\mainmatter  % start of an individual contribution

\title{Krotkodystansowy Komiwojazer \\ Bottleneck Traveling Salesman Problem}

% a short form should be given in case it is too long for the running head
\titlerunning{Bottleneck TSP}

\author{Jakub Ciecierski \and Wojcik Lukasz \\ 
\textit{Warsaw University of Technology, \\
Faculty of Mathematics and Information Science.}}
%

\authorrunning{Bottleneck TSP}

\toctitle{Bottleneck TSP}
\tocauthor{Methodology}
\maketitle





%---------------------------------------------------------------------
%---------------------------------------------------------------------
%---------------------------------------------------------------------
\section{Introduction}


%---------------------------------------------------------------------
%---------------------------------------------------------------------
%---------------------------------------------------------------------
\section{Preliminaries}

\subsection{Triangle inequality}

The cost function c satisfies the \textbf{triangle inequality} if, for all vertices $u,v,w \in V$

\begin{center}
	$c(u,v)\leq c(u,v) + c(v,w)$
\end{center}

It means that adding intermediate steps between some vertices will never decrease the cost.
%---------------------------------------------------------------------
%---------------------------------------------------------------------
\subsection{Approximation Ratio}

%---------------------------------------------------------------------
%---------------------------------------------------------------------
\subsection{TSP}

\subsection{Definition}
\textbf{Traveling salesman problem} in a given graph is asking for a Hamiltonian cycle with the smallest possible sum of weights.

\subsection{Aproximation algorithm}

To find TSP cycle we have to find a \textbf{minimum spanning tree (MST)} first. Sum of weights of MST gives us a lower bound on cost of optimal TSP tour. It also will be used to create a tour whose cost is no more than twice that of the minimum spanning tree's wight (having that it satisfies the triangle inequality). We assume that an input for our algorithm is a complete, undirected graph with vertex r as a root and c satisfies the triangle inequality.


\begin{algorithm}[H]
    \SetKwInOut{Input}{Input}
    \SetKwInOut{Output}{Output}
    \Input{complete, undirected graph $G$, cost function $c$ }
    \Output{Hamiltonian cycle H}
    select a vertex $r \in G.V$ to be a root vertex \\
    
    compute a minimum spanning tree T for G from root r using MST-PRIM(G,c,r) \\
    
    let H be a list of vertices, ordered according to when they are first visited in a preorder tree walk of T \\
    
    return the hamiltonian cycle H \\
    
  %  \underline{function HILL-CLIMBING-IMPROVED} $(s, n)$\;
  %  $i = 0$ \\
  %  $current = s$ \\
  %  \While{$i < n$}
  %  {
  %  	$neighbor$ $\gets$ randomly generated state, different than $current$ \\
    	
    %	\If{OBJ-FUNCTION[neighbor] $\geq$ OBJ-FUNCTION[current]}
    %	{
	%		$current$ $\gets$ $neighbor$
   	%	}
   		  		
   % }
    \caption{Approx-TSP-Tour}
    \label{algo:hc_improved}
\end{algorithm}


%---------------------------------------------------------------------
%---------------------------------------------------------------------
\subsection{Bottleneck TSP}


\begin{algorithm}[h]
    \SetKwInOut{Input}{Input}
    \SetKwInOut{Output}{Output}
    \Input{
        $G(V,E)$ - Connected, undirected graph, where $V$ is the set of vertices and $E$ is the set of edges \\
        $c$ - Cost function. $c: E -> N$
    }
    \Output{$T$ - Minimum Bottleneck Spanning Tree}
    
    \If{$|G.E|$ = 1}{return $G$}
    
    $m = computeMedianWeight(G)$ \\
    $A = \{e : {e {\in G.E}} \land c(e) > m \}$. Set of edges with weight greater than $m$ \\
    $B = \{e : {e{\in G.E}} \land c(e) \leq m \}$. Set of edges with weight smaller or equal to $m$ \\
    $F =$ Forest(B) \\
    
    \If{$|F|=0$}{
        TODO
    }    
    \ElseIf{$|F|=1$}{
        $G^{'}.V = G.V$ \\
        $G^{'}.E = B$ \\
        return MBST$(G^{'})$
    }
    \Else {
        $G^{'}$ = MBST-Contract$(F, A)$
        
        return $F$ $\cup$ MBST($G^{'}$)
    }
    
    \caption{MBST(G, c)}
    \label{alg:mbst}
\end{algorithm}

\begin{algorithm}[h]
    \SetKwInOut{Input}{Input}
    \SetKwInOut{Output}{Output}
    \Input{
        $F$ - Forest,
        $A$ - Set of edges
    }
    \Output{$G$ - Connected graph}
  
    $V_{F} = $ A single super vertex is created for each connected component of $F$. Each such vertex corresponds to its connected component. \\
    $V = $ Missing Vertices in $F$. Vertices in $A$ but not in $F$ \\
    
    $E_{F}$ = Edges from $A$ that connect the connected components of $F$\\
    $E$ = Edges from $A$ to missing vertices in $F$ \\
    
    $G.V = $ $V_{F} \cup V $ \\
    $G.E = $ $E_{F} \cup E $
    
    return $G$

    \caption{MBST-Contract(F, A)}
    \label{alg:mbst_contract}
\end{algorithm}

\begin{algorithm}[h]
    \SetKwInOut{Input}{Input}
    \SetKwInOut{Output}{Output}
    \Input{
        $G(V,E)$ - complete, undirected graph, where $V$ is the set of vertices and $E$ is the set of edges \\
        $r$ - root $r \in G.V$ \\
        $c$ - Cost function satisfying triangle inequality. $c: E -> N$
    }
    \Output{$H$ - Hamiltonian cycle}
    $T = $ MBST$(G,c)$
    \caption{BTSP-Approx(G,c)}
    \label{alg:btsp}
\end{algorithm}

%---------------------------------------------------------------------
%---------------------------------------------------------------------
%---------------------------------------------------------------------
\section{Algorithm}

%---------------------------------------------------------------------
%---------------------------------------------------------------------
%---------------------------------------------------------------------
\section{Analysis}

%---------------------------------------------------------------------
%---------------------------------------------------------------------
\subsection{Correctness}

%---------------------------------------------------------------------
%---------------------------------------------------------------------
\subsection{Time Complexity}


\begin{thebibliography}{4}

\bibitem{mc_rao}


\end{thebibliography}

\end{document}
